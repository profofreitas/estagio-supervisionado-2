\documentclass[a4paper,10pt]{article}
\usepackage[utf8]{inputenc}
%%%%%%%%%%%%%%%%%%%%%%%%%%%%%%%%%%%%%%%%%%%%%%%%%%%%%%%%%%%%%%%%%%%%%%%%%%%%%%%
%% 
\usepackage[a4paper,left=3.0cm,right=2.0cm,top=3.0cm]{geometry}
%%%%%%%%%%%%%%%%%%%%%%%%%%%%%%%%%%%%%%%%%%%%%%%%%%%%%%%%%%%%%%%%%%%%%%%%%%%%%%%

\begin{document}

  \begin{titlepage}
    \begin{center}
      {\large INSTITUTO FEDERAL DE EDUCAÇÃO, CIÊNCIA E TECNOLOGIA } \\[0.2cm]
      {\large CAMPUS TERESINA CENTRAL} \\[0.2cm]
      {\large DEPARTAMENTO DE FORMAÇÃO DE PROFESSORES} \\[0.2cm]
      {\large CURSO DE LICENCIATURA EM FÍSICA} \\[0.2cm]
      {\large DISCIPLINA: ESTÁGIO SUPERVISIONADO II – 100H} \\[0.2cm]
      {\large PROFESSORA: VILANI VASCONCELOS} \\[5.0cm]

      RELATO DA EXPERIÊNCIA \\[0.5cm]
      ESCOLA CAMPO DE ESTÁGIO ESCOLA MUNICIPAL AMBIENTAL 15 DE OUTUBRO\\[2.0cm]
    \end{center}
    
    \begin{flushright}
      Acadêmico(a): Francisco das Chagas Alves Freitas$^{1}$ \\[5.0cm]
    \end{flushright}
      
    \begin{center}
      TERESINA/PIAUÍ \\
      2019.2
    \end{center}
    
  \end{titlepage}
  %   Objetivo do Estágio Supervisionado II, linhas gerais a identificação do estágio, as atividades desenvolvidas, 
  %   algumas aprendizagens e considerações finais
  \textbf{Resumo:} A disciplina Estágio Supervisionado II tem por objetivo a vivência do contexto escolar como um todo e da sala 
  de aula em particular, com vistas a focalizar questões relacionadas a aspectos pedagógicos e ao processo de ensino aprendizagem 
  da área de formação visando uma prática significativa. O estágio foi realizado na Escola Municipal Ambiental 16 de Outubro 
  em algumas turmas dos anos finais do ensino fundamental durante o segundo semestre de 2019. Durante o estágio foram realizados 
  planejamentos de aula e atividades pedagógicas, regência em sala de aula e apoio educacional aos alunos da escola. Durante todo 
  o período do estágio foi notável a dificuldade em manter o interesse dos alunos e elevar o nível de ensino. Mesmo com todo o apoio
  dos professores de gestão escolar, fatores externos no que diz respeito à esfera familiar e mesmo a rede de ensino contribuem 
  para as dificuldades citadas.
  
  \textbf{Palavras-chave:} 
  
%   Descrever em linhas gerais a identificação do estágio, dos orientadores e supervisores, a carga horária cumprida, o objetivo e a 
%   justificativa; informações relevantes da escola (identificação, localização, história, relevância desta para a comunidade onde 
%   está inserida); Apresentar fotos. Apresentar as seções do trabalho.

  \section{Introdução}
    Como parte complementar da discplina Estágio Supervisionado II, parte das horas (60 horas) foram investidas na prática de regência em
    sala de aula, realizada na Escola Municipal Ambiental 15 de Outubro. 
    
    A disciplina tem por objetivo a vivência do contexto escolar como um todo e da sala de aula em particular, com vistas a focalizar questões 
    relacionadas a aspectos pedagógicos e ao processo de ensino aprendizagem da área de formação visando uma prática significativa.
  
    A E. M. Ambiental 15 de Outubro, anteriormente estava localizada na Av. Duque de Caxias, 3470, no bairro Primavera na cidade de Teresina-PI. 
    Por motivos de reforma na sua estrutura física, a escola está utilizando o mesmo endereço da Unidade Escolar Governador Miguel Rosa, localizada 
    na Rua Frei Segismundo, 4º Gre, bairro Pirajá. Sob orientação da \textbf{Professora Célia Barros} e supervisão do \textbf{Professor Edilson}
    
    Inaugurada em 14 de Outubro de 1983, seu nome é uma homenagem ao dia do professor. Inicialmente funcionava somente com as quatro séries 
    iniciais do Ensino Fundamental no turno diurno. Em 1994, implantaram a 5ª e 6ª séries e gradativamente foram implantadas a 7ª e 8ª séries. Em 
    1996 implantou-se o EJA no turno noturno. Em 1993, através do decreto no2404 de 11/08 de 1993 seu nome mudou para Unidade Escolar Ambiental 
    15 de Outubro. 
    
    Em 2008, o Conselho Municipal de Educação através da resolução nº014 de 2008 o funcionamento dos cursos de Ensino 
    Fundamental do 1o ao 9o ano diurno e EJA do 1o ao 4º o bloco noturno novamente com o nome de Escola Municipal Ambiental 15 de Outubro 
    homologado por Silvio Mendes de Oliveira Filho através do decreto nº8004 de 28/11/2008. 
    
    A escola oferece, além do ensino formal, atividades sócio-educativas. Desde 2010 foi instituido o programa Mais Educação. A escola 
    fundamenta-se numa gestão compartilhada composta por dois diretores, dois pedagogos, corpo administrativo, corpo docente, corpo discente e a 
    comunidade. O processo de escolha dos gestores é realizado através de eleição direta com voto universal de maioria simples para um mandato 
    atualmente de três anos.
    
    
  \section{As Atividades Desenvolvidas}
%   GESTÃO DE CLASSE
% - Como organizava o tempo e do espaço da aula;
% - Que recursos/estratégias utilizou que promoveu aprendizagens significativas;
% - Maneiras você adotava para manter a atenção dos alunos durante a aula;
% - Como você resolvia os problemas que surgem durante a aula;
% Que estratégias utilizava para sanar ou minimizar problemas de aprendizagem em sala de aula.
  \subsection{Gestão de Classe}

% SEQUÊNCIA DIDÁTICA. Como fazia o tratamento didático do conteúdo.
% - Introdução (apresentação dos objetivos da aula, levantamento dos conhecimentos prévios dos alunos sobre o tema a
% ser trabalhado; associação dos conhecimentos dos alunos com o assunto da aula e outros). Procedimentos avaliativos: tipos e instrumentos.
% -Desenvolvimento (apresentação dos conteúdos, tipos de atividades realizadas, tratamento didático dado ao erro,
% caracterização dos procedimentos metodológicos utilizados para ensinar os conteúdos selecionados). Procedimentos avaliativos: tipos e instrumentos.
% - Conclusão (síntese). Procedimentos avaliativos: tipos e instrumentos.
  \subsection{Sequência Didática}

  \section{Algumas Aprendizagens}
%   Tomando como parâmetro a sequência didática apresentada acima, relatando as atuações inovadoras implementadas e os resultados alcançados, 
%   explicitando e justificando teoricamente (literatura) os aspectos que foram modificados na prática pedagógica das aulas ministradas, a 
%   partir da situação problema constatada durante o estágio.
  
  \section{Considerações Finais}
%   Descrever sobre os aspectos abaixo fazendo uma reflexão sobre a realidade encontrada (TODO)
% com a literatura da área;
% - Aspectos administrativos e físicos da escola campo;
% - Atividades e resultados que proporcionaram situações inovadoras – com sucesso;
% - Processo de construção dos saberes experiênciais;
% - Saberes desenvolvidos;
% - Comportamentos de aluno ou da classe que despertaram sua atenção;
% - Processo de construção do conhecimento;
% - Processo avaliativo da aprendizagem considerado satisfatório;
% Dúvidas e certezas que surgiram;
% - Ética e equilíbrio emocional em situações conflitantes;
% - O que deu e o que não deu certo? Por quê?
% - Contribuição da escola-campo de estágio para sua formação profissional;
% - Necessidade de investimentos no processo de ensino-aprendizagem (recursos didáticos,
% infraestrutura, formação de pessoal...).
  \newpage
  \section{Referências}
  
    \begin{itemize}
      \item SCALABRIN, Izabel Cristina; MOLINARI, Adriana Maria Corder. A IMPORTÂNCIA DA PRATICA DO ESTÁGIO SUPERVISIONADO NAS 
      LICENCIATURAS. Revista Unar, vol 7, nº 1. 2013;

      \item SILVA, Raimundo Paulino. A Escola Enquanto Espaço de Construção do Conhecimento. Revista Espaço Acadêmico. No139. 
      Dezembro de 2012;
    \end{itemize}

\end{document}
